\section*{好整數計數的數學推導}

令一個正整數 $X$ 為「好整數」,當且僅當存在正整數 $a,b$ 使得
\[
X = 2^a \, b^2,\quad a\ge1,\;b\ge1.
\]

\subsection*{1. 指數拆分}

將指數 $a$ 分為「奇數」和「偶數」兩種情形:

\begin{enumerate}
  \item 若 $a$ 為奇數,寫作 $a=2m+1$,則
    \[
      X = 2^{2m+1}\,b^2
        = 2\,(2^m b)^2
        = 2\,t^2,
      \quad t = 2^m b.
    \]
    這一類好整數皆可寫成 $2\,t^2$。

  \item 若 $a$ 為偶數,寫作 $a=2m$(且因 $a\ge1$,故 $m\ge1$),則
    \[
      X = 2^{2m}\,b^2
        = 4\,(2^{\,m-1}b)^2
        = 4\,t^2,
      \quad t = 2^{\,m-1}b.
    \]
    這一類好整數皆可寫成 $4\,t^2$。
\end{enumerate}

注意 $\{2\,t^2\}$ 與 $\{4\,t^2\}$ 兩集合不重疊:若同時
\[
2\,t^2 = 4\,s^2
\quad\Longrightarrow\quad
t^2 = 2\,s^2,
\]
則 $(t/s)^2=2$,在整數範疇不可能成立。

\subsection*{2. 分別計數}

要統計所有 $X\le N$ 的好整數,只需分別計算以下兩個集合的大小,然後相加:

\[
\begin{aligned}
  &\{\,X=2\,t^2 : 2\,t^2\le N\}
    \;\Longrightarrow\;
    t^2 \le \frac{N}{2}
    \;\Longrightarrow\;
    t \le \sqrt{\frac{N}{2}}
    \;\Longrightarrow\;
    \bigl\lfloor\sqrt{\tfrac{N}{2}}\bigr\rfloor, \\[6pt]
  &\{\,X=4\,t^2 : 4\,t^2\le N\}
    \;\Longrightarrow\;
    t^2 \le \frac{N}{4}
    \;\Longrightarrow\;
    t \le \sqrt{\frac{N}{4}}
    \;\Longrightarrow\;
    \bigl\lfloor\sqrt{\tfrac{N}{4}}\bigr\rfloor.
\end{aligned}
\]

因此,好整數的總數為
\[
\#\{X\le N\}
= \Bigl\lfloor\sqrt{\tfrac{N}{2}}\Bigr\rfloor
+ \Bigl\lfloor\sqrt{\tfrac{N}{4}}\Bigr\rfloor.
\]

\subsection*{3. 程式實現}

在 C++ 中,用
\begin{verbatim}
(ll) sqrt((long double)N/2)
\end{verbatim}
等價於 $\lfloor\sqrt{N/2}\rfloor$,同理
\verb|(ll) sqrt((long double)N/4)| 等價於 $\lfloor\sqrt{N/4}\rfloor$。  
將兩者相加,即可在 $O(1)$ 時間內得到答案。
